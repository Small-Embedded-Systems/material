%% -*- mode: latex; mode:flyspell -*-
\documentclass[svgnames,x11names]{beamer}

\usepackage[british]{babel}

\usepackage{minted,calc,siunitx}
\usepackage{tikz}
\usetikzlibrary{chains,positioning,calc,shadows,arrows,matrix}
\usetikzlibrary{shapes.geometric,shapes.symbols}
\usetikzlibrary{circuits,circuits.logic,circuits.logic.IEC}

\usepackage{pgfplots}
\usepackage{booktabs,inconsolata}

\usepackage[theorems,skins,minted]{tcolorbox}


\title{C -- complex declarations}
\subtitle{CM0506 -- Small Embedded Systems}
\date{Lecture 4b}
\author{Dr Alun Moon}
\institute{Department of Computer and Information Science}

\definecolor{NUblue}{RGB}{62,141,165}
\definecolor{NUbluedark}{RGB}{40,119,143}

\usetheme{CambridgeUS}

\usecolortheme{crane}
\setbeamercolor*{palette primary}{use=structure,fg=white,bg=NUblue}
\setbeamercolor*{palette quaternary}{fg=white,bg=NUbluedark}
\setbeamercolor{section in head/foot}{fg=white,bg=NUbluedark}
\setbeamercolor{subsection in head/foot}{fg=white,bg=NUblue}
\setbeamercolor{frametitle}{fg=NUbluedark!150,bg=NUblue!40}
\setbeamercolor{title in head/foot}{fg=white,bg=NUblue}
\setbeamercolor{author in head/foot}{fg=white, bg=NUbluedark}
\setbeamercolor{date in head/foot}{fg=white, bg=NUblue!60}
\setbeamercolor{title}{fg=NUbluedark!150,bg=NUblue!30}
\setbeamercolor{date}{fg=NUbluedark!150}
\setbeamercolor{block title}{fg=white,bg=NUblue}

\usepackage[T1]{fontenc}
\usepackage[utf8]{inputenc}

\begin{document}

\frame\maketitle

\begin{frame}[fragile]{C's complex declarations}
The syntax of C can become bewildering, especially in the more complex declarations involving pointers.   
  \vspace{1em}

\begin{tabular}{ll}\toprule
  \mintinline{c}{int n;} & $n$ is an integer \\
  \mintinline{c}{int *q;} & $q$ is a pointer to an integer\\
  \mintinline{c}{int s[];} & $s$ is an array of integers\\
  \mintinline{c}{int f();} & $f()$ is a function returning an
                             integer\\
  \mintinline{c}{int *g();} & $g()$ is a function returning a pointer
                              to an integer\\
  \mintinline{c}{int (*h)();} & $h$ in a pointer-to-a-function
                               returning an integer\\\bottomrule
\end{tabular}
\end{frame}

\begin{frame}[fragile,fragile]
  \frametitle{Arrays and Pointers}
  Arrays (of integers) and Pointers (to integers) are (almost) interchangable
  \begin{block}{}
\begin{minted}{c}
    int n, m, o;
    int *a;
    int k[5];

    a = k;
    n = a[2];
\end{minted}
  \end{block}
  Arrays are often passed to functions and returned from functions as
  pointers
  \begin{block}{}
\begin{minted}{c}
    int data[100];
    int *residuals;

    residuals = variance(data);
    residuals[4];
\end{minted}
  \end{block}
\end{frame}

\begin{frame}[fragile]
  \frametitle{Functions and pointers}
  Pointers to functions allow functions to be passed as parameters
  \begin{block}{}
\begin{minted}{c}
int f(int x)
{
    return x*x;
}
int n=6, p,q;
int (*g)(int);

p = f(n);
g = f;
q = g(n);
plot(1, 5, f ); /* assume function to plot */
\end{minted}
  \end{block}
\end{frame}

% \begin{frame}
%   \frametitle{C standard library}
%   \framesubtitle{sort and search}
% The standard C library provides utility functions that sort and search
% data in arrays.
%   \begin{block}{}
% \begin{minted}{c}
% struct command {
%   char *name;
%   char *(*action)(char*);
% \end{minted}
%   \end{block}
% \end{frame}
%%%% --------

\end{document}
%% Local Variables:
%% mode: reftex
%% mode: auto-fill
%% mode: flyspell
%% End:
