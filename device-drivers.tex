%% -*- mode: latex; mode:flyspell -*-
\documentclass[svgnames,x11names]{beamer}

\usepackage[british]{babel}

\usepackage{minted,tikz,tcolorbox,calc,siunitx}
\usepackage{tikz-timing}
\usetikzlibrary{chains,positioning,shapes.geometric,shapes.symbols,calc,shadows}
\usetikzlibrary{circuits.ee.IEC,arrows}
\usepackage{pgfplots}
\usepackage{booktabs,inconsolata}


\title{The Device Driver}
\subtitle{CM0506 -- Small Embedded Systems}
\date{Lecture 3}
\author{Dr Alun Moon}
\institute{Department of Computer and Information Science}

\definecolor{NUblue}{RGB}{62,141,165}
\definecolor{NUbluedark}{RGB}{40,119,143}

\usetheme{CambridgeUS}

\usecolortheme{crane}
\setbeamercolor*{palette primary}{use=structure,fg=white,bg=NUblue}
\setbeamercolor*{palette quaternary}{fg=white,bg=NUbluedark}
\setbeamercolor{section in head/foot}{fg=white,bg=NUbluedark}
\setbeamercolor{subsection in head/foot}{fg=white,bg=NUblue}
\setbeamercolor{frametitle}{fg=NUbluedark!150,bg=NUblue!40}
\setbeamercolor{title in head/foot}{fg=white,bg=NUblue}
\setbeamercolor{author in head/foot}{fg=white, bg=NUbluedark}
\setbeamercolor{date in head/foot}{fg=white, bg=NUblue!60}
\setbeamercolor{title}{fg=NUbluedark!150,bg=NUblue!30}
\setbeamercolor{date}{fg=NUbluedark!150}
\setbeamercolor{block title}{fg=white,bg=NUblue}

\usepackage[T1]{fontenc}
\usepackage[utf8]{inputenc}

\begin{document}

\frame\maketitle

\begin{frame}{Interpretation of Hardware Specifications}
Typical hardware specifications include:
\begin{itemize}
\item Functional description
\item Pinout specifications
\item Operating voltages (Minimum, maximum, and typical)
\item Timing Diagrams
\item Protocol Diagrams
\item Critical timing data
\end{itemize}
\end{frame}

\begin{frame}{What is a \alert{device driver}?}
  \begin{itemize}
  \item A collection of software routines to perform I/O functions
  \item Interface software, called by the operating system or
    application code, to configure devices and perform I/O
  \item Software to \emph{glue} the hardware and software together
  \item Separates policy from mechanism
  \end{itemize}
\end{frame}

\begin{frame}{A Device Driver}
  \begin{itemize}
  \item Encapsulates the behaviour of a device
  \item Allows application developers to ignore low-level detail
  \item A consistent interface to a device or family of devices
  \end{itemize}
\end{frame}

\begin{frame}{Device Driver code}
  \begin{itemize}
  \item Notoriously difficult to design and debug
  \item May be complex
  \item Requires a deep understanding of the hardware
  \item Low-level code -- sometimes requires assembly language
  \item API (Application Programming Interface) requires careful design
  \end{itemize}
\end{frame}

\begin{frame}{Portability}
  \begin{itemize}
  \item Device drivers provide a layer of abstraction to hardware I/O devices
  \item Higher levels of software can access devices in a uniform
    hardware-independent manner
  \item If designed well, device driver software can be ported.
  \end{itemize}
\end{frame}

\begin{frame}{Developing Device Drivers}
  \begin{itemize}
  \item Read the hardware specification
  \item Re-read the specification, review in a group
  \item Specify an API and review this
  \item Design and develop code to provide the API and consistent with
    hardware specifications
  \item Test the API carefully -- use instumentation, and simple,
    incremental, text harness software.
  \end{itemize}
\end{frame}

\begin{frame}{Typical Driver Functions}
  \begin{itemize}
  \item Configure a device -- initialise the hardware to a known state
  \item Turn a device on or off
  \item Assign interrupt handlers
  \item Read data from a device
  \item Write data to a device
  \end{itemize}
\end{frame}

\begin{frame}{Timing diagrams}
  \begin{table}
    \begin{tabular}{rl}\toprule
      Interrupt 0      & \texttiming{L2{G12L}GL} \\
      LED              & \texttiming{L{12H12L}H} \\ \midrule
      Interrupt 1      & \texttiming{L4{G6L}GL} \\
      LED              & \texttiming{L2{6H6L}H} \\ \bottomrule
    \end{tabular}
    \caption{Timing diagram for two LEDs}
    \label{tab:twoleds}
  \end{table}
\end{frame}

\begin{frame}[fragile]
\def\degr#1{\makebox[2em][r]{#1}\ensuremath{{}^{\circ }}}%
\begin{tikztimingtable}[%
  timing/dslope=0.1,timing/.style={x=2ex,y=2ex},x=2ex,
  timing/rowdist=3ex,
  timing/name/.style={font=\sffamily\scriptsize},
  timing/nodes/advanced,
]
  Clock \degr{90} & l 2{C} N(A1) 5{C} \\
  Clock \degr{180}& C 2{C} N(A2) 4{C} c\\
  Clock \degr{270}& h 3{C} N(A3) 4{C} \\
  Clock \degr{0}  & [C] 2{C} N(A0) 2{C} N(A4) 3{C}c ;[ dotted ]
                    2L; 2{C} N(A5) 3{C} \\
  Start of T$_{\text{sw }}$ & 4S G N( start ) \\
  Input Pulse & 2.0 L 2H 3.5L ;[ dotted ] 2L; 5L \\
  Set 3 & 2.5 L 2H 3.0 L ;[ dotted ] 2L; 5L \\
  Set 2 & 3.0 L 2H 2.5 L ;[ dotted ] 2L; 5L \\
  Set 1 & 3.5 L 2H 2.0 L ;[ dotted ] 2L; 5L \\
  Set 0 & 4.0 L 2H 1.5 L ;[ dotted ] 2L; 5L \\
  Reset & 7.5 L ;[ dotted ] 2L; 2L N( reset ) 2H 1L \\
  Final Pulse & 2.5 L N(B1) e N(B2) e N(B3) e 3.5H; [ dotted ]
      2H; 2H 3L \\
  Counter & N(x) 2D{ Full } N(B0) 2D {0} N(B4) 2D{1} 1.5 D;[ dotted ]
  .25 D {2} 1.75 D {};
  2D{~ D$_\text{M}$} N(B5) 2D{ D$_\text{M}$ +1} D N(y )\\
  \extracode
  %\ tablegrid
  \node[anchor=north] at ($ (x) ! .5 ! (y) - (0 ,.75) $)
    {\scriptsize D$_\text{M}$ = MSBs of Duty Cycle };
  \end{tikztimingtable}%
\end{frame}
%%%% --------
\end{document}
%% Local Variables:
%% mode: reftex
%% mode: auto-fill
%% mode: flyspell
