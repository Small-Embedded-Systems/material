%% -*- mode: latex; mode:flyspell -*-
\documentclass[svgnames,x11names]{beamer}

\usepackage[british]{babel}

\usepackage{minted,tikz,tcolorbox,calc,siunitx}
\usepackage{tikz-timing}
\usetikztiminglibrary{either}
\usetikztiminglibrary{counters}
\usetikztiminglibrary{beamer}
\usetikzlibrary{chains,positioning,shapes.geometric,shapes.symbols,calc,shadows}
\usetikzlibrary{circuits.ee.IEC,arrows}
\usepackage{pgfplots}
\usepackage{booktabs,inconsolata}


\title{The Device Driver}
\subtitle{CM0506 -- Small Embedded Systems}
\date{Lecture 3}
\author{Dr Alun Moon}
\institute[CIS]{Department of Computer and Information Science}

\definecolor{NUblue}{RGB}{62,141,165}
\definecolor{NUbluedark}{RGB}{40,119,143}

\usetheme[secheader]{Boadilla}
%\usetheme{CambridgeUS}
%\usecolortheme{crane}
\setbeamercolor*{palette primary}{use=structure,fg=white,bg=NUblue}
\setbeamercolor*{palette quaternary}{fg=white,bg=NUbluedark}
\setbeamercolor{section in head/foot}{fg=white,bg=NUbluedark}
\setbeamercolor{subsection in head/foot}{fg=white,bg=NUblue}
\setbeamercolor{frametitle}{fg=NUbluedark!150,bg=NUblue!40}
\setbeamercolor{title in head/foot}{fg=white,bg=NUblue}
\setbeamercolor{author in head/foot}{fg=white, bg=NUbluedark}
\setbeamercolor{date in head/foot}{fg=white, bg=NUblue!60}
\setbeamercolor{title}{fg=NUbluedark!150,bg=NUblue!30}
\setbeamercolor{date}{fg=NUbluedark!150}
\setbeamercolor{block title}{fg=white,bg=NUblue}

\usepackage[T1]{fontenc}
\usepackage[utf8]{inputenc}

\begin{document}

\frame\maketitle

\begin{frame}{Interpretation of Hardware Specifications}
Typical hardware specifications include:
\begin{itemize}
\item Functional description
\item Pinout specifications
\item Operating voltages (Minimum, maximum, and typical)
\item Timing Diagrams
\item Protocol Diagrams
\item Critical timing data
\end{itemize}
\end{frame}

\begin{frame}{What is a \alert{device driver}?}
  \begin{itemize}
  \item A collection of software routines to perform I/O functions
  \item Interface software, called by the operating system or
    application code, to configure devices and perform I/O
  \item Software to \emph{glue} the hardware and software together
  \item Separates policy from mechanism
  \end{itemize}
\end{frame}

\begin{frame}{A Device Driver}
  \begin{itemize}
  \item Encapsulates the behaviour of a device
  \item Allows application developers to ignore low-level detail
  \item A consistent interface to a device or family of devices
  \end{itemize}
\end{frame}

\begin{frame}{Device Driver code}
  \begin{itemize}
  \item Notoriously difficult to design and debug
  \item May be complex
  \item Requires a deep understanding of the hardware
  \item Low-level code -- sometimes requires assembly language
  \item API (Application Programming Interface) requires careful design
  \end{itemize}
\end{frame}

\begin{frame}{Portability}
  \begin{itemize}
  \item Device drivers provide a layer of abstraction to hardware I/O devices
  \item Higher levels of software can access devices in a uniform
    hardware-independent manner
  \item If designed well, device driver software can be ported.
  \end{itemize}
\end{frame}

\begin{frame}{Developing Device Drivers}
  \begin{itemize}
  \item Read the hardware specification
  \item Re-read the specification, review in a group
  \item Specify an API and review this
  \item Design and develop code to provide the API and consistent with
    hardware specifications
  \item Test the API carefully -- use instumentation, and simple,
    incremental, text harness software.
  \end{itemize}
\end{frame}

\begin{frame}{Typical Driver Functions}
  \begin{itemize}
  \item Configure a device -- initialise the hardware to a known state
  \item Turn a device on or off
  \item Assign interrupt handlers
  \item Read data from a device
  \item Write data to a device
  \end{itemize}
\end{frame}

\begin{frame}{Timing diagrams}
  \begin{table}
    \begin{tabular}{rl}\toprule
      Interrupt 0      & \texttiming{L2{G12L}GL} \\
      LED              & \texttiming{L{12H12L}H} \\ \midrule
      Interrupt 1      & \texttiming{L4{G6L}GL} \\
      LED              & \texttiming{L2{6H6L}H} \\ \bottomrule
    \end{tabular}
    \caption{Timing diagram for two LEDs}
    \label{tab:twoleds}
  \end{table}
\end{frame}

\begin{frame}[fragile]{I$^2$C bus}
  A communications bus using 2-wires 
  \begin{description}
  \item[SCL] clock signal
  \item[SDA] data signal
  \end{description}

  \begin{tikztimingtable}[xscale=1.5,
    timing/dslope=0.25,timing/lslope=0.25,
    timing/new counter={char=B}]
    SDA & H[dotted];LL8{2B}LlH;H[dotted] \\
    SCL & H[dotted];hH8{LH}LlHh;H[dotted] \\
    \extracode
    \begin{background}
    \uncover<2>{
      \foreach \b in {0,...,7} \fill[blue!25] (\b *2+3.75,1.5) rectangle +(.75,-4);
      \fill[green!25] (1.25,1.4) rectangle +(1.25,-4);
      \fill[red!25] (20.75,1.4) rectangle +(0.75,-4);
    }
    \end{background}
  \end{tikztimingtable}

  The state of the bus is given by the following conditions
  \begin{tabular}{lrr}\toprule
    State & SCL & SDA \\ \midrule
    Start condition & H & L \\
    Data valid & H & bit \\
    Start condition & H & H \\
  \end{tabular}
\end{frame}

\begin{frame}{SPI}
  different
\end{frame}
%%%% --------
\end{document}
%% Local Variables:
%% mode: reftex
%% mode: auto-fill
%% mode: flyspell
