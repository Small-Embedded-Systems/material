%% -*- latex -*-
\documentclass[12pt]{tufte-handout}

\usepackage[british]{babel}
\usepackage{booktabs}
\usepackage{tikz}
\usetikzlibrary{calc,shapes.geometric} 
\usepackage{tikz-timing}
\usepackage{minted}
\usepackage{graphicx}
\usepackage{natbib}
\usepackage{siunitx}
\usepackage[theorems,skins]{tcolorbox}
\tcbset{enhanced}
\newtcbtheorem{exercise}{Exercise}{drop fuzzy shadow}{ex}
\newtcbtheorem{question}{Question}{drop fuzzy shadow}{q}

\title{LPC4088 IO Connections\\Devices on the Quick Start board and
  Experiment Base board}
\author{Dr Alun Moon}

\definecolor{code}{wave}{602}
%\definecolor{cmd}{wave}{528}
\definecolor{cmd}{named}{SkyBlue}

\hypersetup{colorlinks, urlcolor = DarkRed}


\begin{document}
\maketitle

\section{LEDs}
The LPC4088 Quickstart board has 4 LEDs.  The two green LEDs (LED1 and
LED2) are connected as part of the USB interface, thoough they can be
used independently via GPIO pins.  The two blue LEDs (LED3 and LED4)
are connected to GPIO pins.

The Experiment Base board has 11 LEDs.  Three are housed in the same
package as one component on the board.  These are a \emph{red},
\emph{green}, and \emph{blue} LED connected to GPIO pins.  These can also
be driven by PWM outputs from the LPC4088.  

The other 8 LEDs are connected to the outputs of a
\emph{shift-register} which is connected to the LPC4088 via a SPI bus
(under SSP in the LPC User manual).
\begin{table}
\begin{tabular}{llrrcc}
  Board & LED & Port & pin & active & pwm \\
  \midrule[1pt]
  Quick start & {LED1} (green) & P1 & 18 & low  & n \\
              & {LED2} (green) & P0 & 13 & low & n \\
              & {LED3} (blue)  & P1 & 13 & high & n \\
              & {LED4} (blue)  & P2 & 19 & high & n \\
 \midrule
  Experiment  & {LED1 - red}   & P1 & 11 & low & y \\
              & {LED1 - green} & P1 &  5 & low & y \\
              & {LED1 - blue}  & P1 &  7 & low & y \\
 \midrule
   Experiment & {LED2} -- {LED9} & \multicolumn{4}{c}{Shift
  Register --- (\emph{see below})} \\ \bottomrule
\end{tabular}
\caption{{led} connections}
\label{tab:led-connections}
\end{table}
\subsection{Shift register}
Eight  of the {LED}s on the Experiment Base Board are connected
to the outputs of a shift register.  The shift register itself is
connected to the SPI bus as follows:
\begin{table}
\begin{tabular}{lrr}
  SPI & Port & pin \\ \midrule
  MOSI & P1 & 24 \\
  SCK & P1 & 20 \\
  SSEL & P1 & 2 \\ \bottomrule
\end{tabular}
\caption{Shift register SPI bus connections}
\label{tab:spishiftregister}
\end{table}
\bibliography{lpc4088}
\bibliographystyle{plainnat}
\end{document}

%% Local Variables:
%% mode: reftex
%% mode: auto-fill
%% mode: flyspell
%% mode: tabkey2
%% End:
