%% -*- latex -*-
\documentclass[a4paper]{tufte-handout}

\usepackage[british]{babel}
\usepackage{booktabs}
\usepackage{tikz}
\usetikzlibrary{shapes.geometric}
\usepackage{tikz-timing}
\usepackage{minted}
\usepackage{graphicx}
\usepackage{natbib}
\usepackage{siunitx}
\usepackage[theorems,skins]{tcolorbox}
\tcbset{enhanced,title filled}
\newtcbtheorem{exercise}{Exercise}{drop fuzzy shadow}{ex}
\newtcbtheorem{question}{Question}{drop fuzzy shadow}{q}
\usepackage{manfnt}
\hypersetup{colorlinks,urlcolor=DarkRed}
\usepackage{stackengine}
\usepackage{scalerel}

\title{Starting Embedded C Programming\\\small{CM0506 Small Embedded Systems}}
\author{Dr Alun Moon}
\date{Seminar 1}

\definecolor{code}{wave}{602}
%\definecolor{cmd}{wave}{528}
\definecolor{cmd}{named}{SkyBlue}

\graphicspath{{screenshots/}}
\begin{document}
\maketitle
\setlength{\parskip}{0.5em}
\setlength{\parindent}{0pt}

\newcommand\dangersign[1][2ex]{%
  \renewcommand\stacktype{L}%
  \scaleto{\stackon[1.3pt]{\color{red}$\triangle$}{\tiny!}}{#1}%
}

\begin{abstract}
  This exercise will introduce you to using the development
  environment to compile, build, downnload, and debug code on the LPC4088
  used in S2.  Along the way you will see some of the subtle
  differences between Java and C.  The biggest difference is not one
  between C and Java, but between code run on the PC (Java) and code
  run in an embedded environment.
\end{abstract}
\section{Edit -- Save -- Build -- Load -- Debug, cycle}
Unlike the PC environment where you can make changes to your code,
re-run it, and see the effects.  Embedded systems programming has more
steps, each critical, but easy to miss coming from something like
BlueJ.

The principle steps are:
\begin{description}
\item[Edit] make changes to your code
\item[Save] save the changes
\item[Build] (re)build the program, this itself has several steps:
  \begin{description}
  \item[Compile] the C code into processor instructions
  \item[Assemble] any low level assembler code into processor
    instructions
  \item[Link] the binaries together, resolving address of storage and
    function calls
  \end{description}
\item[Load] the completed program into the memory of the target device
  (LPC4088)
\item[Debug] Check the code for correct operation, using the debugger
  to inspect the code execution and contents of variables.  Identify
  the causes of any incorrect behavior.
\end{description}

\section{GitHub}
\newthought{For this module we will be using \emph{Git} version
  control software and \emph{GitHub} as a sofware repository.}

For this exercise you  will need the code found at
\url{https://github.com/dr-alun-moon/embeddedC} (see figure
\ref{fig:github}.  For now use the option to download a zip-file of
the archived code.  

\begin{figure}
  \flushright
  \includegraphics[width=\textwidth]{web-download}
  \caption{GitHub repository}
  \label{fig:github}
\end{figure}

Once you have the code downloaded, extract the files from the
zip-file\marginnote[-4em]{\begin{tcolorbox}[colframe=Red]
    \dangersign[4ex]    It won't work if you just view the files, they \emph{must} be
    unpacked.
  \end{tcolorbox}}.  Once you have them on the disk you
should see something like figure \ref{fig:unpacked}.
\begin{figure}[h]
  \flushright
  \includegraphics[width=\textwidth]{Unpacked}
  \caption{Unpacked project}
  \label{fig:unpacked}
\end{figure}

Start the Keil {$\mu$Vision} IDE from the start menu (figure \ref{fig:uvision}).
\begin{figure}[h]
  \flushright
  \includegraphics{keil}
  \caption{Keil $\mu$Vision}
  \label{fig:uvision}
\end{figure}
Then use the ``Project Open'' menu to open the project you unpack in
the previous step (figure \ref{fig:open-project}).
\begin{figure}[h]
  \flushright
  \includegraphics[width=\textwidth]{open-project}
  \caption{Open project}
  \label{fig:open-project}
\end{figure}
You should now have something looking like figure \ref{fig:ide}.
\begin{figure}
  \includegraphics[width=\textwidth]{open-ide}
  \caption{Open IDE}
  \label{fig:ide}
\end{figure}
\clearpage
\section{C for embedded systems}
\newthought{Take a moment to examine the program.}\sidenote{reproduced
  in listing \ref{lst:simple}}
You should see something familiar to you from your Java experience.

Variable declarations, control flow and other constructs should look
familiar.  One thing to note is that variable declarations should
always occur at the beginning of a code block (the thing between
braces \texttt{\{} \ldots \texttt{\}}), or be global.  The other key
feature that applies to embedded systems is that \emph{the program
  does not exit!} (see line 17 in listing \ref{lst:simple}).

\subsection{The Build process}
The process of building and running your code is simple, with several steps:
\paragraph{Build the program (figure \ref{fig:build}).}  The IDE includes a clever tool that
builds the target program.  By monitoring \emph{which} files in the
project have changed, it can work out what in just-enough to do, to
rebuild the program.  You'll see that it performs several steps on the
first build, and less on subsequent builds.
  \begin{figure}
    \flushright
    \includegraphics[width=\textwidth]{build}
    \caption{Build project}
    \label{fig:build}
  \end{figure}
\begin{tcolorbox}[colframe=red!50!black]
Watch the ``Build Output'' window at the bottom of the IDE for progress.
\end{tcolorbox}
\clearpage
\paragraph{Download code to flash memory (figure \ref{fig:load}).}  Once build (if successful)
the program cannot be run yet.  It is still in the PC (as a file on
the disk)\sidenote{It cannot be run from here.  It is built for the
  instruction set of the ARM processor, not the instruction set of the
  PC (AMD x86 instruction set)}.   It has to be downloaded into the
Flash memory of the LPC4088.
  \begin{figure}
    \flushright
    \includegraphics[width=\textwidth]{load}
    \caption{Load into embedded system's flash memory}
    \label{fig:load}
  \end{figure}
Once loaded into the memory the program can be run.

\begin{exercise}{Run the code}{run}
  The simplest (but unexciting) way of running the code is to press
  the reset button.  Found just above the USB connector.

  \includegraphics{reset-button}

  What can you see happening as a result of the program running?
\end{exercise}

\clearpage
\section{The Debugger}
\paragraph{To see what is happening in the program you need the
  monitor/debugger.}  Start the debugging session (figure
\ref{fig:debug}) using the icon in the IDE
\begin{figure}
  \flushright
  \includegraphics[width=\textwidth]{start-debug}
  \caption{Start debugging session}
  \label{fig:debug}
\end{figure}
Once the debugger session starts you will have a window something like
figure \ref{fig:the-debugger}.
\begin{figure}
  \includegraphics[width=\textwidth]{debug-window}
  \caption{The Debugger}
  \label{fig:the-debugger}
\end{figure}
The debugger allows you to 
\begin{itemize}
\item examine the values of variables
\item control execution of the code
  \begin{itemize}
  \item step line-by-line through the program
  \item set breakpoints, points where the program will be halted for
    inspection
  \end{itemize}
\item examine CPU registers
\item examine memory content
\end{itemize}

\subsection{Watching variables}
In the program listing if you hover the mouse over a
variable, a tool-tip shown its current value.
\newthought{If you have critical variables in the code} you can have a
permanent view in the \emph{watch} list.
If you \emph{Right click on a variable,} and from the menu you can add
the variable to a watch list (figure \ref{fig:add-watch})
\begin{figure}
  \flushright
  \includegraphics[width=\textwidth]{add-watch}
  \caption{add to watch list}
  \label{fig:add-watch}
\end{figure}

\begin{exercise}{Add variables to the watch list}{addwatch}
  Add the variables \texttt{a}, \texttt{b}, \texttt{f}, and \texttt{n}
  to the watch list.
\end{exercise}

The watch list should look something like figure \ref{fig:watch}
\begin{figure}
  \flushright
  \includegraphics[width=\textwidth]{watch-list}
  \caption{Watch list}
  \label{fig:watch}
\end{figure}
\begin{tcolorbox}[colframe=red!50!black]
As you step through the program you will see the variables highlighted
when their value changes
\end{tcolorbox}

\subsection{Controlling execution}
\newthought{The debugger has several ways to control the execution of
  the program under inspection.}  This allows for a fine level of
control, to examine the operation of the code in detail.  

\paragraph{Single stepping the code.}  Executes one line of the
program code at a time (figure \ref{fig:step})
\begin{figure}
  \includegraphics[width=\textwidth]{step}
  \caption{Single step}
  \label{fig:step}
\end{figure}
\begin{exercise}{Step through the code}{stepping}
  Step through the code, watching how the program flow works and
  watching the values of the variables change as you do.
\end{exercise}

\paragraph{Run to cursor.}  Put the cursor on a specific line, and the
``run to cursor'' (figure \ref{fig:runcursor}) will run the program
until it reaches the line containing the cursor.
\begin{figure}
  \includegraphics{run-to-cursor}
  \caption{Run to cursor}
  \label{fig:runcursor}
\end{figure}

\clearpage
\paragraph{Reset}  The reset function resets the program to its starting
point.
\begin{figure}
  \includegraphics{reset}
  \caption{Reset }
  \label{fig:reset}
\end{figure}
\subsection{Breakpoints}
\newthought{Breakpoints are a valuable tool.}  A break point is a
predefined point in the program where execution is halted.
\paragraph{To set a breakpoint} click the mouse pointer in the left
margin, by the line you want the breakpoint at (figure
\ref{fig:set-break}).  A red circle indicates the presence of a breakpoint.
\begin{figure}
  \includegraphics[width=\textwidth]{set-breakpoint}
  \caption{Setting a breakpoint}
  \label{fig:set-break}
\end{figure}
With breakpoints set you can run the code (figure \ref{fig:run}) and
it will run then stop at the breakpoint.
\begin{figure}
  \includegraphics{run}
  \caption{run program}
  \label{fig:run}
\end{figure}
\begin{tcolorbox}[colframe=red!50!black]
In more complex programs, a breakpoint can be a useful diagnostic in
itself.  If a breakpoint is never triggered, you know that the program
\emph{never} reaches that point.  
\end{tcolorbox}

\section{Debugging exercise}
There is a \emph{broken} program at
\url{https://github.com/dr-alun-moon/debuggingEx} 
take a copy of this and see if you can find the programming errors.

\appendix
\begin{listing}
  \inputminted[linenos]{c}{embeddedC/src/main.c}
  \caption{Simple embedded C program}
  \label{lst:simple}
\end{listing}

%\bibliographystyle{plainnat}
%\bibliography{}
\end{document}
\begin{minted}[frame=leftline,framerule=1mm,rulecolor=\color{code}]{C}
	LPC_IOCON->P1_18 = 0;  /* D */
	LPC_GPIO1->DIR |= led1pin;
	LPC_GPIO1->SET = led1pin;
\end{minted}

% min systick 
% 8ns ~ 
% peripheral clock 60MHz => 16ns (8ns half-life of K meson

%% Local Variables:
%% mode: reftex
%% mode: auto-fill
%% mode: flyspell
%% End:
