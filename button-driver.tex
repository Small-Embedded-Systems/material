%% -*- latex -*-
\documentclass[a4paper]{tufte-handout}

\usepackage[british]{babel}
\usepackage{booktabs}
\usepackage{tikz}
\usepackage{tikz-timing}
\usepackage{minted}
\usepackage{graphicx}
\usepackage{natbib}
\usepackage{siunitx}
\usepackage[theorems,skins]{tcolorbox}
\tcbset{enhanced}
\newtcbtheorem{exercise}{Exercise}{drop fuzzy shadow}{ex}
\newtcbtheorem{question}{Question}{drop fuzzy shadow}{q}

\title{LPC4088 Button Device Driver\\\small{CM0506 Small Embedded Systems}}
\author{Dr Alun Moon}
\date{Seminar 5b}
\definecolor{code}{wave}{602}
%\definecolor{cmd}{wave}{528}
\definecolor{cmd}{named}{SkyBlue}

\begin{document}
\maketitle

\section{Button Device Driver}
\newthought{If we take the user board button\sidenote{The one on the
    red PCB}} by itself,  the task of designing an API becomes fairly
simple.  We need to be able to initialise the hardware to have the
button as an input, and we need to be able to read the button state.

\paragraph{The API definitions in the header file} will look something
like.
\begin{minted}[frame=leftline,framerule=1mm,rulecolor=\color{code}]{C}
void button_init(void);
int button_pressed(void);
\end{minted}
With only one button, the API is simply a case of configuring the GPIO
port for input on the appropriate pin, and reading the state of the
pin.

\section{Internals}
\newthought{The internals of the driver merit some examination and
  discussion.}  There are a number of features that the processor has
around the port that are worth looking at in detail as they can make
life easier for the programmer.
\subsection{Initialisation}
\paragraph{For the most part the initialisation of the port is
  familiar.}  The pin is put into its GPIO mode and set for input.
\begin{minted}[frame=leftline,framerule=1mm,rulecolor=\color{code}]{C}
void button_init(void){
  LPC_IOCON->P2_10 = PULLUP|INVERT; /* use pullup resistor and signal inversion */
  LPC_GPIO2->DIR &= ~buttonpin; /* input - clear direction bit */
}
\end{minted}
\newthought{The IOCON register is worth examining, its value is no longer 0.}

The User manual \citep{lpc4088} lists the behaviour of the IOCON
register in Chapter 7.  The button is connected to Port-2 pin-10,
table 79 on page 127 lists this as a type D pin.  Tables 83 and 84
describe this in more detail.

\paragraph{Bits 3:4 are described as MODE bits,} controlling an internal
resistor.  Without going into the details of the electronics.  This
can be set to no resistor (00) or a pull-up resistor (10).  Setting
this as a pull-up resistor ensures that a high value is read if the
input is not connected to another signal.\sidenote{We can get away
  with no resistor, but I prefer to use the pull-up resistor as it is
  available and ensures a \emph{High} or \emph{Low} signal.}
 

\paragraph{Bit 6 controls an inverter \citep[7.3.4]{lpc4088}.}
Setting this bit turns the inversion on.  A signal level of \emph{Low}
on the pin now reads as a logic 1, and a \emph{High} signal now reads
as 0.   This is useful here as the bit now reflects the state of the
button, \emph{pushed}(1), or \emph{not-pushed}(0).
 
\newthought{By setting the ICON register to \texttt{PULLUP|INVERT}.}
We are guaranteed a 0 or 1, with 1 being the button pressed.

\subsection{Reading the button state}
Reading the button state is now a simple matter of testing bit-10 in
port-2.
\begin{minted}[frame=leftline,framerule=1mm,rulecolor=\color{code}]{C}
int button_pressed(void){
  int state = 0;
  LPC_GPIO2->MASK = ~buttonpin;
  state = LPC_GPIO2->PIN;
  LPC_GPIO2->MASK = nil;
  return state;
}
\end{minted}
\newthought{The built-in MASK register} performs a similar action to
the masking operations we have already seen in code.  Its operation is
subtly different.
\paragraph{A value of 1} marks the bit as not to update the value of
the PIN register on read \citep[8.5.1.2]{lpc4088}.

For convenience I read the line
as
\begin{minted}[frame=leftline,framerule=1mm,rulecolor=\color{code}]{C}
  LPC_GPIO2->MASK = ~buttonpin;
\end{minted}
\begin{quote}
  Mask (\emph{hide}) \textbf{not} the button pin.
\end{quote}
When we then read the value of the PIN register, we will get 0s
everywhere \emph{except} for the value of bit-10, which reflects the
state of the button.
\paragraph{Since C treats 0 as false, non-zero as true} and we have a
value of 0 or 1024 ($2^{10}$), then we can just return this
value\sidenote{you could test this and return true/false as required,
  or shift the bits down 10 places to give 0 or 1}.

\clearpage
\section{Interrupts}
\newthought{The User Button is connected to Port-2, which can generate
  interrupts.}  We can add to the API to initialise and configure an
interrupt handler for the button.

\subsection{Rising and Falling edge triggers}
Section 8.2.2 of the manual, says that the port can generate
interrupts on rising and falling edges.  A rising-edge is where the
value goes from 0-to-1, and a falling-edge where the value goes from
1-to-0. 

These values are taken from the PIN register, so after the effects of
the inverter on the pin.  So given the configuration of IOCON we have
used, we get
\begin{table}
  \begin{tabular}{ll}
    Button changes from \emph{not}-pushed to pushed&rising edge\\
    Button changes from pushed to \emph{not}-pushed&falling edge\\
  \end{tabular}
  \caption{Button interrupts}
  \label{tab:buttonedges}
\end{table}

\subsection{Initialisation}
There is one interrupt generated for GPIO registers, IRQ 38.  The
interrupt controller has two registers, one for disabling the
interrupt and one for enabling the interrupt.  Because the number of
interrupts is greater than 32, CMSIS defines an array, which is why
the code for enable/disable looks a little odd.  IRQ 38 is bit 6 in
interrupt register 1.

To select the rising or falling edge to trigger the interrupt, we set
the appropriate bit (10) in the Port interrupt control registers.

\begin{minted}[frame=leftline,framerule=1mm,rulecolor=\color{code}]{C}
NVIC->ICER[1] = (1UL<<(GPIO_IRQn-32)); /* disable interrupt */

LPC_GPIOINT->IO2IntEnR = buttonpin;    /*  rising edge triggers interrupt */
LPC_GPIOINT->IO2IntEnF = buttonpin;    /* falling edge triggers interrupt */
	
NVIC->ISER[1] = (1UL<<(GPIO_IRQn-32)); /* enable interrupt */
\end{minted}

In the interrupt handler \mintinline{c}{GPIO_IRQHandler} we can look
at the status registers for rising and falling edges to identify the
cause of the interrupt.

\paragraph{In the initialisation} of the interrupt handler we pass in
a pointer to a function for the custom code for the application.  If
this function takes the appropriate \mintinline{c}{enum} type (see the
header file), we can pass the type of edge to this when called from
the interrupt handler.
\begin{minted}[frame=leftline,framerule=1mm,rulecolor=\color{code}]{C}
if( LPC_GPIOINT->IO2IntStatR ) button_event(Risingedge);
if( LPC_GPIOINT->IO2IntStatF ) button_event(Fallingedge);
\end{minted}
where
\begin{minted}[frame=leftline,framerule=1mm,rulecolor=\color{code}]{C}
static void (*button_event)(enum edge_t);
\end{minted}

\section{5-way joystick button switch}
The Experiment Base Board, as a 5-way button switch.  This is
connected to port-5\sidenote{If you read the User
  Manual\citep{lpc4088} you will see that port-5 only has 5 pins
  exposed.}

\subsection{Design choice}
In designing the API for the 5-way switch you have 
\bibliographystyle{plainnat}
\bibliography{lpc4088}

\appendix

\end{document}

%% Local Variables:
%% mode: reftex
%% mode: auto-fill
%% mode: flyspell
%% End:
