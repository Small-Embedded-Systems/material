%% -*- latex -*-
\documentclass{beamer}
\usepackage[british]{babel}
\usepackage{minted,inconsolata,tikz}
%\usepackage[svgnames]{xcolor}

\title{CM506 -- Small Embedded Systems}
\subtitle{Best Practice, Style and Code Conformance}
\author{Dr Alun Moon}
\institute[CSDT]{Department of Computer Science and Digital Technology}
\date{Lecture 6}

\definecolor{NUblue}{RGB}{62,141,165}
\definecolor{NUbluedark}{RGB}{40,119,143}

\usetheme{CambridgeUS}

\usecolortheme{crane}
\setbeamercolor*{palette primary}{use=structure,fg=white,bg=NUblue}
\setbeamercolor*{palette quaternary}{fg=white,bg=NUbluedark}
\setbeamercolor{section in head/foot}{fg=white,bg=NUbluedark}
\setbeamercolor{subsection in head/foot}{fg=white,bg=NUblue}
\setbeamercolor{frametitle}{fg=NUbluedark!150,bg=NUblue!40}
\setbeamercolor{title in head/foot}{fg=white,bg=NUblue}
\setbeamercolor{author in head/foot}{fg=white, bg=NUbluedark}
\setbeamercolor{date in head/foot}{fg=white, bg=NUblue!60}
\setbeamercolor{title}{fg=NUbluedark!150,bg=NUblue!30}
\setbeamercolor{date}{fg=NUbluedark!150}
\setbeamercolor{block title}{fg=white,bg=NUblue}
\logo{\vspace*{-0.3cm}{\includegraphics[width=2cm]{NU_logo_blue}}}

\usepackage[T1]{fontenc}
\usepackage[utf8]{inputenc}

\begin{document}
\frame\maketitle


\begin{frame}{Best Practice}
  \begin{itemize}
  \item The adoption of procedures, tools and techniques to
    ensure quality in the final product
  \item Sound engineering practice
  \item Accepted methods
  \end{itemize}
\end{frame}

\begin{frame}{Issues}
  \begin{description}
  \item[Complete specification] you may need to go back and ask questions
  \item[Management of complexity] Good partioning and decomposition
  \item[Design quality]
  \item[Modelling]
  \item[Record keeping]  version control
  \item[Software quality]  attention to detail
  \item[Testing rigour]
  \end{description}
\end{frame}

\begin{frame}{Cost}
  \begin{itemize}
  \item Cost of product recall as a result of product defects
    can be massive
  \item Automobile faults require expensive recall
  \item Even modest systems (e.g., controllers for white goods)
    require that sufficient care has been adopted in their development
    such that they are safe to use, are reliable, are not too
    expensive to run or repair, etc….
  \end{itemize}
\end{frame}

% \begin{frame}{Good Practice}
%   \begin{itemize}
%   \item Quality in implementation and testing
%   \item General issues
%     \begin{itemize}
%     \item Reinvention, time dependencies, control structure,
%       error detection and testing, design and design tools,
%       optimisation, portability, configuration and global variables
%     \end{itemize}
%   \item Language issues
%     \begin{itemize}
%     \item C, safe language subsets, conformance and style
%       checkers
%     \end{itemize}
%   \item Working culture
%   \end{itemize}
% \end{frame}

% \begin{frame}{Some good sources}
%   \begin{itemize}
%   \item Safer C: Developing software for high-integrity and
%     safety-critical systems, L Hatton, McGrawHill, 1994
%   \item C Traps and Pitfalls, A Koenig, Addison Wesley, 1988
%   \item Writing Solid Code, S Maguire, Microsoft Press, 1993
%   \item Code Complete: A practical handbook of software construction,
%     S McConnell, Microsoft Press, 1993
%   \end{itemize}
% \end{frame}

% \begin{frame}{Quality in Implementation}
%   \begin{itemize}
%   \item Was the design correctly translated into C?
%   \item Is the software properly constructed in a modular fashion?
%   \item Is the software written in accordance with a standard such as
%     MISRA-C?
%   \item Is the software source written to a standard style guide?
%   \item Is the software free of any unintended recursion?
%   \end{itemize}
% \end{frame}

\begin{frame}{Quality in Implementation}
  \begin{itemize}
  \item Is the software designed to be easy to test?
%  \item Is the compiler fit for the purpose?
  \item Is the software properly constructed in a modular fashion?
  \item Have the compiler switches and non-ISO C extensions been
    properly used?
  \item Has the target CPU been used properly or to best advantage
    \begin{itemize}
    \item Configuration of target
    \item Clock speed?
    \item Errata sheets - silicon bugs
    \end{itemize}
  \end{itemize}
\end{frame}

\begin{frame}{Quality in Testing}
  \begin{itemize}
    \item Do you have a good test plan?
  \item Are testing tools capable enough or sufficiently
    accurate?
  \item Have timing aspects been considered in test schema?
  \item Has all code been tested under normal and abnormal conditions
    with no bugs discovered
  \item Have all failed tests been explained and corrected
    \begin{itemize}
    \item Have all tests been performed whilst running on the
      target hardware?
    \end{itemize}
  \item Have any differences between the hardware used for testing and
    the final hardware been allowed for?
  \item Have tests been performed using the final memory map?
  \end{itemize}
\end{frame}

\begin{frame}{Quality in Testing}
  \begin{itemize}
%  \item Have all tests been performed on exactly the same
%    software as will be shipped?
%  \item Have all required modifications been incorporated into the
%    released version?

  \item Has all the source code been statically analysed?
  \item Has the complexity of the code been assessed?
  \item Have interrupts been considered?
  \item Are interrupt run times been considered?
  \item Are interrupt latencies within acceptable limits?
  \end{itemize}
\end{frame}

\begin{frame}{Do not reinvent}
  \begin{itemize}
  \item Look for similar problems
  \item Use templates - design patterns
  \item Learn from experience
  \item Focus on similarities not differences
  \item Take care using code not designed for reuse
  \item Is your code designed for reuse?  A personal library?
  \end{itemize}
\end{frame}

\begin{frame}[fragile]{Time dependencies}
\begin{itemize}
\item Do not build these into software
\item Do not implement delays as empty loops
\item Use operating system system clocks
\item Use timers with polling or interrupts
  \begin{alertblock}{do not use}
\begin{minted}{c}
    for ( i= 0; i < 1000 ; i++ ) ;
\end{minted}
  \end{alertblock}
  \begin{exampleblock}{good practice}
\begin{minted}{c}
    initialiseCounter( 1000 );
    waitForTimeout();
\end{minted}
  \end{exampleblock}
\end{itemize}
\end{frame}

\begin{frame}{Structure}
\begin{itemize}
\item Don’t use large if..then..else structures
\item Use Boolean algebra,
\item Use lookup tables
\item Redesign as state-machine

\item Do not use large switch statements
%\item Do not assume that switch is simpler than if..then..else
%\item Same clauses at if..then..else
\item State machine gives switch structure
\end{itemize}
\end{frame}

\begin{frame}{Recursion}
  \begin{itemize}
  \item Recursion is a powerful design paradigm
  \item Many algorithms can be implemented elegantly using recursion
  \item Recursion can cause stack overflow!
  \item Can be very difficult to follow processing chain in recursive
    algorithms
  \item Many recursive algorithms can be replaced by iteration
  \end{itemize}
\end{frame}


\begin{frame}{Error detection and Testing}
\begin{itemize}
\item Testing
\begin{itemize}
\item Prepare proper test schemas
\item Avoid interactive testing
\item Prepare complete test beds
\item Test complete applications in addition to isolated
components
\end{itemize}
\item Run-time error detection
\begin{itemize}
\item Design error detection and handling into your design
\item Do not add error handling as an after thought
\item Do not discover error detection requirements through
testing
\end{itemize}
\end{itemize}
\end{frame}


\begin{frame}{Design}
  \begin{itemize}
  \item Do not over design
  \item Include nothing useless

  \item Co-design software and hardware
  \item Some functions can more easily be implemented in hardware
  \item Software can be engineered to make hardware requirements
    simpler
  \item Consider alternative solutions
  \item The best solution is unlikely to be the first
  \item Present your designs to others

  \item Use a design method
  \item But understand its semantics and notation and syntax
  \end{itemize}
\end{frame}

% \begin{frame}{Design Tools}
% \begin{itemize}
% \item Do not choose on the basis pf market fashion
% \begin{itemize}
% \item Evaluate technical needs
% \item Do not believe what vendors say
% \end{itemize}
% \item Do not document after implementation
% \item Use documentation interactively and iteratively as
%  part of the design process
% \begin{itemize}
% \item Understand the different roles of documentation
% \item Elicitation
% \item Specification
% \item Detailed design
% \item Maintenance navigation
% \end{itemize}
% \end{itemize}
% \end{frame}

% \begin{frame}{Analysis}
%   \begin{itemize}
%   \item Do scheduling analysis
%     \begin{itemize}
%     \item Include device drivers in analysis
%     \item Include interrupts in analysis
%     \end{itemize}
%   \item Do memory analysis
%   \item Also stack usage

%   \item Use modelling whenever possible
%   \item Better than testing
%   \item Promotes early design

%   \item Use formal methods if appropriate
%   \item Encourages rigour
%   \end{itemize}
% \end{frame}

\begin{frame}{Optimisation}
\begin{itemize}
\item A focus on optimisation can confuse design
\item Use selective optimisation
\item Profilers are useful tools
\item Call graphs
\item Localise hot spots
\item Be aware of your target hardware and compilers
\item \emph{Measure} don't guess
\end{itemize}
\end{frame}


\begin{frame}{Portability}
\begin{itemize}
\item Different architectures confound porting
\item Localise design and implementation volatility
\item Encapsulate all non standard features
\item Separate machine dependent and machine dependent code
\item Do not over generalise everything
\item Take care using code not designed for reuse
\end{itemize}
\end{frame}

\begin{frame}{Program Configuration}
\begin{itemize}
\item Avoid \#defines for configuration
\item The use of pragmas should be minimised, localised and
  encapsulated to dedicated functions
\item Use data files as the preferred method for configuration
\item Change control and multiple versions are easier
\end{itemize}
\end{frame}

\begin{frame}{Globals}
  \begin{itemize}
  \item Minimise the use of global variables
  \item Global variables are shared…
  \item Thread violations
  \item Increases functional coupling
  \item Data inconsistency
  \item Causes inter-module dependencies
  \item Limits replication of modules
  \end{itemize}
\end{frame}

% \begin{frame}{Global Variables}
%   \begin{itemize}
%   \item ADT programming
%     \begin{itemize}
%     \item Encapsulation
%     \item Defined interface
%     \item Use locks and scheduling to access shared data
%     \end{itemize}
%   \item If non OO technology is used:
%   \item Use a struct to encapsulate
%   \item Pass pointers to structures as args
%   \item Use initialisation functions
%   \end{itemize}
% \end{frame}

\begin{frame}{C}
  \begin{itemize}
  \item C has poor run time checking
  \item Easy to make mistakes with C syntax (= or ==, etc.)
  \item Widely used for embedded development - a highlevel assembler
  \item C is very mature - good compilers, known problems.
  \item C is very portable and compilers generate small code.
  \item Can use a restricted language (subset) for improved control:
  \item MISRA is used for automotive applications
  \end{itemize}
\end{frame}

\begin{frame}{Style conventions}
  \begin{itemize}
  \item Names - length, type, etc
  \item Layout - braces, indent, etc
  \item Headers - files for prototypes
  \item Function declaration
  \item Commenting - how and what depth
    \begin{itemize}
    \item Indian Hill
    \item CodeStyle - Linux kernel style - Linus Torvalds
    \item GNU
    \end{itemize}
  \end{itemize}
\end{frame}

%\part{Making Good % Use of The C Programming Language}

% \begin{frame}{C Data Management}
%   \begin{description}
%   \item[Type] - char, unsigned int, ...
%   \item[Identifiers] - used to access data
%   \item[Values] - the data stored at a location
%   \item[Scope] - the range of code lines over which the identifier is
%     recognised
%   \item[Lifetime] - from creation to destruction - when available for
%     use
%   \end{description}
% \end{frame}

% \begin{frame}{Scope}
  
% \end{frame}

% \begin{frame}{Globals}
%   \begin{itemize}
%   \item Must be declared outside all functions
%   \item Functions are inherently global cannot declare a function
%     inside a function
%   \item Scope is from the point of declaration to the end of the file
%   \end{itemize}
% \end{frame}

% \begin{frame}{Lifetime}
%   \begin{itemize}
%   \item Static - Life of entire program execution
%   \item Automatic - From entry to exit of a function - stored on the
%     stack during function execution
%   \item Dynamic - from malloc() to free() stored on the heap
%   \end{itemize}
% \end{frame}

\end{document}

%% Local Variables:
%% mode: flyspell
%% mode: reftex
%% mode: buffer-face
%% End: