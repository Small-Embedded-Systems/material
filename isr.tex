%% -*- latex -*-
\documentclass[12pt]{tufte-handout}

\usepackage[british]{babel}
\usepackage{booktabs}
\usepackage{tikz}
\usepackage{tikz-timing}
\usepackage{minted}
\usepackage{graphicx}
\usepackage{natbib}
\usepackage{siunitx}
\usepackage[theorems,skins]{tcolorbox}
\tcbset{enhanced}
\newtcbtheorem{exercise}{Exercise}{drop fuzzy shadow}{ex}
\newtcbtheorem{question}{Question}{drop fuzzy shadow}{q}

\title{Interrupt Service Routines\\\small{CM0506 Small Embedded Systems}}
\author{Dr Alun Moon}

\definecolor{code}{wave}{602}
%\definecolor{cmd}{wave}{528}
\definecolor{cmd}{named}{SkyBlue}


\begin{document}
\maketitle
\section{Interrupt Service Routines (ISR)}
{Interrupt Service Routines (ISR) are unlike normal
  functions.}  Normal functions have an entry-point that is called by
other functions (with parameters), and an exit-point that returns
control (and a value) to the calling routine.  The point in a program
where a function is called is known and predicable.
An ISR by its nature departs from these patterns.
\newthought{An ISR:}
\begin{itemize}
\item Takes no parameters passed into it.
\item Returns no values
\item Can occur at any (random) time -- is \emph{asynchronous}
\end{itemize}
The prototype, as seen in a header file for an ISR is:-
\begin{minted}[frame=leftline,framerule=1mm,rulecolor=\color{code}]{c}
void someISR(void);
\end{minted}
\subsection{Code size}
An ISR can occur at any time, and most likely happens between critical
operations\sidenote{For example between the \emph{increment} and \emph{test} of a
  \texttt{for} loop}, there is a need to keep the time the ISR takes
to execute to a minimum.
\newthought{In event driven systems,} most of the behaviour can occur
in interrupts.  While the execution time may not be as so constrained,
care must be taken that the execution-time of the ISR is known in
relation to the time available for it to execute.

If the ISR takes too long to execute, the behaviour of the system can
be compromised.  \emph{Priority-inversion} can occur where a low
priority ISR can block a higher-priority interrupt from occuring.
Nested interrupts and interrupt masks can help aleviate symptoms here.  
If the ISR takes too long to handle high-frequency interrupts, then
successive interrupts may not be processed.  

\newthought{It follows that slow operations,} such as drawing to the
display, should be avoided in ISRs.
  
\subsection{Communications}
The ISR takes no parameters and returns no values.  This makes
communicating with the rest of the program difficult.  The only way to
pass information from the IST to other parts of the program is to use
\emph{global variables}.  This introduces many potential problems so
extreme care is needed.  Good use of \texttt{static} and scoping rules
restricts access to only those parts that need it.  Ideally use 
variables that can be updated in a single (assembly) operation.

There may be other cases, such as an ISR handling communications via
serial, ethernet, or CAN.  In these cases a shared buffer can be used,
and care taken to avoid the problems raised by simultaneous access of
producer and consumer parts of the program.

\begin{exercise}{Git download of initial code}{initial}
\begin{minted}[frame=leftline,framerule=1mm,rulecolor=\color{cmd}]{bash}
$ git
\end{minted}
%$
\end{exercise}
\begin{question}{enum and switch}{switch}
\begin{tcolorbox}[colframe=red!50!black,title=Solution]
\end{tcolorbox}
\end{question}

\bibliographystyle{plainnat}
\bibliography{lpc4088}

\clearpage
\appendix

\end{document}

%% Local Variables:
%% mode: reftex
%% mode: auto-fill
%% mode: flyspell
%% End: