%% -*- latex -*-
\documentclass[a4paper]{tufte-handout}

\usepackage[british]{babel}
\usepackage{booktabs}
\usepackage{tikz}
\usepackage{tikz-timing}
\usepackage{minted}
\usepackage{graphicx}
\usepackage{natbib}
\usepackage{siunitx}
\usepackage[theorems,skins]{tcolorbox}
\tcbset{enhanced}
\newtcbtheorem{exercise}{Exercise}{drop fuzzy shadow}{ex}
\newtcbtheorem{question}{Question}{drop fuzzy shadow}{q}

\title{LPC4088 Button Interrupts\\\small{CM0506 Small Embedded Systems}}
\author{Dr Alun Moon}
\date{Seminar 4a}

\definecolor{code}{wave}{602}
%\definecolor{cmd}{wave}{528}
\definecolor{cmd}{named}{SkyBlue}

\begin{document}
\maketitle
\newthought{Now we have got the hang of Timer interrupts,}  we can
turn our attention to interrupts generated by the push button.

\section{Interrupt generation}
Looking at the processor manual \citep[section 8.2.2, pg144]{lpc4088}
GPIO~Ports 0 and 2 can generate inputs.  From the schematic
\citep{quickstart} we can see that the user button in connected to
Port 2 pin 10.  From the Experiment Base-board schematic
\citep{baseboard} 
\begin{minted}[frame=leftline,framerule=1mm,rulecolor=\color{code}]{C}
enum LED {
 LED1, LED2, LED3, LED4,
 left_green=LED1, right_green,
 left_blue, right_blue
};
\end{minted}
The functions to control the LEDs take this as a parameter,
\begin{minted}[frame=leftline,framerule=1mm,rulecolor=\color{code}]{C}
ledOn(enum LED name);
ledOff(enum LED name);
ledToggle(enum LED name);
int ledState(enum LED name);
\end{minted}

These, along with the prototype for \verb'ledInit()' can be put in the
header \path{led.h}.


\begin{exercise}{Git download of initial code}{initial}
    Retrieve the project from the \textsc{Git} repository.
\begin{minted}[frame=leftline,framerule=1mm,rulecolor=\color{cmd}]{bash}
$ git clone https://github.com/dr-alun-moon/timers
$ cd timers
$ git checkout ex.1.1
\end{minted}
    Examine the files for the LED driver \path{led.h} and
    \path{led.c}.
    \begin{enumerate}
    \item Can you follow the way the code is structured?
    \item Why are the SET and CLR registers used to turn different
      LEDs on?
    \end{enumerate}
  \end{exercise}

\bibliographystyle{plainnat}
\bibliography{lpc4088}

\end{document}

% min systick 
% 8ns ~ 
% peripheral clock 60MHz => 16ns (8ns half-life of K meson
% max 1/60MHz 22 =~ 307Gs ~ 9742.197785461 siderial years
%% Local Variables:
%% mode: reftex
%% mode: auto-fill
%% mode: flyspell
%% End:
