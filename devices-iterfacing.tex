%% -*- mode: latex; mode:flyspell -*-
\documentclass[svgnames,x11names]{beamer}

\usepackage[british]{babel}

\usepackage{minted,tikz,tcolorbox,calc,siunitx}
\usetikzlibrary{chains,positioning,calc,shadows,arrows,matrix}
\usetikzlibrary{shapes.geometric,shapes.symbols}
\usetikzlibrary{circuits,circuits.logic,circuits.logic.IEC}

\usepackage{pgfplots}
\usepackage{booktabs,inconsolata}


\title{Sensors and Actuators}
\subtitle{CM0506 -- Small Embedded Systems}
\date{Lecture 4a}
\author{Dr Alun Moon}
\institute{Department of Computer and Information Science}

\definecolor{NUblue}{RGB}{62,141,165}
\definecolor{NUbluedark}{RGB}{40,119,143}

\usetheme{CambridgeUS}

\usecolortheme{crane}
\setbeamercolor*{palette primary}{use=structure,fg=white,bg=NUblue}
\setbeamercolor*{palette quaternary}{fg=white,bg=NUbluedark}
\setbeamercolor{section in head/foot}{fg=white,bg=NUbluedark}
\setbeamercolor{subsection in head/foot}{fg=white,bg=NUblue}
\setbeamercolor{frametitle}{fg=NUbluedark!150,bg=NUblue!40}
\setbeamercolor{title in head/foot}{fg=white,bg=NUblue}
\setbeamercolor{author in head/foot}{fg=white, bg=NUbluedark}
\setbeamercolor{date in head/foot}{fg=white, bg=NUblue!60}
\setbeamercolor{title}{fg=NUbluedark!150,bg=NUblue!30}
\setbeamercolor{date}{fg=NUbluedark!150}
\setbeamercolor{block title}{fg=white,bg=NUblue}

\usepackage[T1]{fontenc}
\usepackage[utf8]{inputenc}

\begin{document}

\frame\maketitle

\begin{frame}{What are They?}{Actuators}
  \begin{itemize}
  \item Change something in the environment 
  \item Under computer control
  \item  Provide feedback to a process to maintain the predetermined
    set-points 
  \item  Convert electrical signals into physical changes
  \end{itemize}
\end{frame}
\begin{frame}{What are They?}{Sensors}
  \begin{itemize}
  \item Allow sampling of an environment or process
    variable 
  \item Convert changes in the process under control into
    electrical signals
  \end{itemize}

\end{frame}

\begin{frame}{Sensors}{Physical quantities measurable:}
\begin{itemize}
\item Temperature
\item Pressure
\item Distance/displacement
\item Velocity
\item Acceleration
\item Fluid flow
\item Light Intensity
\item Voltage
\item Current
\item Resistance
\item Force
\item Liquid level
\item Torque
\item pH
\item + many more...
\end{itemize}
\end{frame}

\begin{frame}{Switches}
\begin{itemize}
\item Binary devices
\item Normally closed/normally open
\item Momentary closure
\item Position - closure of contacts - confirm position of
mechanical parts
\item Human hands - keyboards, etc.
\item Conditions - pressure, temperature (thermostats)
\item Compressed air switches
\item Hydraulics
\end{itemize}

\end{frame}

\begin{frame}{Pressure measurement}
\begin{itemize}
\item Pressure bends an elastic element such as a
diaphragm, tube, bellows or piston
\item The displacement in turn moves a needle,
change an electrical impedance or
resistance
\item Piezoelectric pressure sensors
\item Rapid changes in pressure are difficult to
measure - why?
\item High pressure transducers are costly
\end{itemize}
\end{frame}

\begin{frame}{Temperature}
\begin{itemize}
\item Expansion of solids, liquids or gases
\item Pressure or movement changes can be
measured
\item Thermocouples - junction between
dissimilar metals - generate small voltages
\item Other solid state devices are available such
as thermistors
\end{itemize}
\end{frame}

\begin{frame}{Actuators}
\begin{description}
\item[Relays] switch large electrical loads using
a small electrical input
\item[Solenoids] Motion of an iron core in a coil
does mechanical work
\item[Motors] perform mechanical work
  \begin{itemize}
  \item rotation or other through levers 
  \item DC and Stepper
  \end{itemize}
\end{description}
\end{frame}

\begin{frame}{Relays}
\begin{itemize}
\item Power switching devices
\item Were used in large numbers by PO
\item Small current and voltage inputs
\item Allow switching of large currents/voltages
\item Switch AC or DC
\item Multiple contacts - changeover
\item Electrical isolation
\item Reed relays are sealed in a glass tube
\item Opto-isolators - solid state devices - electrical
isolation of components
\end{itemize}
\end{frame}

\begin{frame}{Motors}
\begin{itemize}
\item DC motors are employed very widely in
industry, appliances, automobiles, etc.
\item Used to provide continuous rotation or no
rotation - position
\item Inexpensive and efficient
\item Can use PWM for speed control - noisy
\item Geared for more torque
\end{itemize}
\end{frame}
\begin{frame}{Stepper Motors}
\begin{itemize}
\item Digitally controlled
\item Discrete positioning
\item Useful where accurate control is required
\item Lower torque than DC
\item Pulses cause the motor to rotate in steps -
perhaps 1.8$\deg$
 per pulse
\item Positional feedback is not required (unless
the motor slips)
\end{itemize}
\end{frame}

\begin{frame}{Accuracy}
\begin{description}
\item[Accuracy] the total of all deviations between a measured
value and the actual value - sum of non-linearity,
repeatability and hysteresis.
\item[Non-linearity] the maximum difference in measured value
or output from a straight line between calibration points
\item[Repeatability] the max difference in a measured value or
output when a set point is approached multiple times from
above or below
\item[Hysteresis] the max difference in measured value or
output when a set value is approached from above, and then
below the value.
\end{description}
\end{frame}

\begin{frame}{Signal Conditioning}
\begin{itemize}
\item Usually some manipulation of the signals is
required between a computer interface and
sensors/actuators
\item Change in power - power levels in computer
interfaces are low
\item Voltage levels - e.g., 0-5 v to -2.5-2.5
\item Current to voltage and voltage to current
\item Requires use of operational amplifiers, etc.
\end{itemize}
\end{frame}

\part{Ananlogue to Digital Conversion}
\frame\partpage

\begin{frame}{ADC}
{Data acquisition equipment}
  \begin{enumerate}
  \item receives an analogue signal
   
    
  \item it is converted to a voltage

    
  \item in order to transfer to the computer/processor that voltage needs
    to be \alert{digitised} by an Analogue-to-Digital Converter (ADC)
  \end{enumerate}

\end{frame}
%%%% --------
\end{document}
%% Local Variables:
%% mode: reftex
%% mode: auto-fill
%% mode: flyspell
%% End:
